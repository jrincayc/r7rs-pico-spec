%\vfill\eject
\chapter{Basic concepts}
\label{basicchapter}

\section{Variables, syntactic keywords, and regions}
\label{specialformsection}
\label{variablesection}

An identifier\mainindex{identifier} can name either a type of syntax or
a value.  An identifier that names a type
of syntax is called a {\em syntactic keyword}\mainindex{syntactic keyword}
and is said to be {\em bound} to a transformer for that syntax.  An identifier that names a
value is called a {\em variable}\mainindex{variable} and is said to be
{\em bound} to that value.  The set of all visible
bindings\mainindex{binding} in effect at some point in a program is
known as the {\em environment} in effect at that point.  The value
to which a variable is bound is called the
variable's value.  In \rsevenrs{} variables are technically bound to
a memory location instead of a value. Pico Scheme implementations may
bind variables to values.

\vest Certain expression types bind variables to values.
These expression types are called {\em binding constructs}.
\mainindex{binding construct}

The most fundamental of the variable binding constructs is the
{\cf lambda} expression, because all other variable binding constructs
can be explained in terms of {\cf lambda} expressions.  The other
variable binding constructs are {\cf let} and {\cf define}.

%Hm, is define fundamental?

\vest Scheme is a language with
block structure.  To each place where an identifier is bound in a program
there corresponds a \defining{region} of the program text within which
the binding is visible.  The region is determined by the particular
binding construct that establishes the binding; if the binding is
established by a {\cf lambda} expression, for example, then its region
is the entire {\cf lambda} expression.  Every mention of an identifier
refers to the binding of the identifier that established the
innermost of the regions containing the use.  If there is no binding of
the identifier whose region contains the use, then the use refers to the
binding for the variable in the global environment, if any
(chapters~\ref{expressionchapter} and \ref{initialenv}); if there is no
binding for the identifier,
it is said to be \defining{unbound}.\mainindex{bound}\mainindex{global
environment}

\section{Disjointness of types}
\label{disjointness}

No object satisfies more than one of the following predicates:

\begin{scheme}
boolean?          null?
number?           pair?
procedure?        symbol?
\end{scheme}

These predicates define the types 
{\em boolean}, the empty list object,
{\em number, pair, procedure}, and {\em symbol}.
\mainindex{type}\schindex{boolean?}\schindex{pair?}\schindex{symbol?}
\schindex{number?}\schindex{procedure?}\mainindex{empty list}

Although there is a separate boolean type,
any Scheme value can be used as a boolean value for the purpose of a
conditional test.  As explained in section~\ref{booleansection}, all
values count as true in such a test except for \schfalse{}.
This report uses the word ``true'' to refer to \schtrue{} and any
Scheme value except \schfalse{}, and the word ``false'' to refer to
\schfalse{}. \mainindex{true} \mainindex{false}

\section{External representations}
\label{externalreps}

An important concept in Scheme (and Lisp) is that of the {\em external
representation} of an object as a sequence of characters.  For example,
an external representation of the integer 28 is the sequence of
characters ``{\tt 28}'', and an external representation of a list consisting
of the integers 8 and 13 is the sequence of characters ``{\tt(8 13)}''.

The external representation of an object is not necessarily unique.  The
integer 28 also has a representation ``{\tt +28}'', and the
list in the previous paragraph also has the representations ``{\tt( 08 13
)}'' and ``{\tt(8 .\ (13 .\ ()))}'' (see section~\ref{listsection}).

Many objects have standard external representations, but some, such as
procedures, do not have standard representations (although particular
implementations may define representations for them).

An external representation can be written in a program to obtain the
corresponding object (see {\cf quote}, section~\ref{quote}).

Note that the sequence of characters ``{\tt(+ 2 6)}'' is {\em not} an
external representation of the integer 8, even though it {\em is} an
expression evaluating to the integer 8; rather, it is an external
representation of a three-element list, the elements of which are the symbol
{\tt +} and the integers 2 and 6.  Scheme's syntax has the property that
any sequence of characters that is an expression is also the external
representation of some object.  This can lead to confusion, since it is not always
obvious out of context whether a given sequence of characters is
intended to denote data or program, but it is also a source of power,
since it facilitates writing programs such as interpreters and
compilers that treat programs as data (or vice versa).

The syntax of external representations of various kinds of objects
accompanies the description of the primitives for manipulating the
objects in the appropriate sections of chapter~\ref{initialenv}.

\section{Storage model}
\label{storagemodel}

Since side effects are not allowed, implementations may choose to
use any convenient storage model.

\section{Proper tail recursion}
\label{proper tail recursion}

Implementations of Scheme are required to be
{\em properly tail-recursive}\mainindex{proper tail recursion}.
Procedure calls that occur in certain syntactic
contexts defined below are {\em tail calls}.  A Scheme implementation is
properly tail-recursive if it supports an unbounded number of active
tail calls.  A call is {\em active} if the called procedure might still
return.  Calls can
return at most once and the active calls are those that have not
yet returned.
A formal definition of proper tail recursion can be found
in~\cite{propertailrecursion}.

\begin{rationale}

Intuitively, no space is needed for an active tail call because the
continuation that is used in the tail call has the same semantics as the
continuation passed to the procedure containing the call.  Although an improper
implementation might use a new continuation in the call, a return
to this new continuation would be followed immediately by a return
to the continuation passed to the procedure.  A properly tail-recursive
implementation returns to that continuation directly.

Proper tail recursion was one of the central ideas in Steele and
Sussman's original version of Scheme.  Their first Scheme interpreter
implemented both functions and actors.  Control flow was expressed using
actors, which differed from functions in that they passed their results
on to another actor instead of returning to a caller.  In the terminology
of this section, each actor finished with a tail call to another actor.

Steele and Sussman later observed that in their interpreter the code
for dealing with actors was identical to that for functions and thus
there was no need to include both in the language.

\end{rationale}

A {\em tail call}\mainindex{tail call} is a procedure call that occurs
in a {\em tail context}.  Tail contexts are defined inductively.  Note
that a tail context is always determined with respect to a particular lambda
expression.

\begin{itemize}
\item The last expression within the body of a lambda expression,
  shown as \hyper{tail expression} below, occurs in a tail context.
\begin{grammar}%
(l\=ambda \meta{formals}
  \>\arbno{\meta{definition}} \meta{tail expression})

\end{grammar}%

\item If one of the following expressions is in a tail context,
then the subexpressions shown as \meta{tail expression} are in a tail context.
These were derived from rules in the grammar given in
chapter~\ref{formalchapter} by replacing some occurrences of \meta{body}
with \meta{tail body},  and some occurrences of \meta{expression}
with \meta{tail expression}.  Only those rules that contain tail contexts
are shown here.

\begin{grammar}%
(if \meta{expression} \meta{tail expression} \meta{tail expression})
(if \meta{expression} \meta{tail expression})

(cond \atleastone{\meta{cond clause}})
(cond \arbno{\meta{cond clause}} (else \meta{tail expression}))

(and \arbno{\meta{expression}} \meta{tail expression})
(or \arbno{\meta{expression}} \meta{tail expression})

(let (\arbno{\meta{binding spec}}) \meta{tail body})

{\rm where}

\meta{cond clause} \: (\meta{test} \meta{tail expression})

\meta{tail body} \: \arbno{\meta{definition}} \meta{tail expression}
\end{grammar}%


\end{itemize}

In addition, the first argument passed to \ide{apply} must be called
via a tail call.

In the following example the only tail call is the call to {\cf f}.
None of the calls to {\cf g} or {\cf h} are tail calls.  The reference to
{\cf x} is in a tail context, but it is not a call and thus is not a
tail call.
\begin{scheme}%
(lambda ()
  (if (g)
      (let ((x (h)))
        x)
      (and (g) (f))))
\end{scheme}%

\begin{note}
Implementations may
recognize that some non-tail calls, such as the call to {\cf h}
above, can be evaluated as though they were tail calls.
In the example above, the {\cf let} expression could be compiled
as a tail call to {\cf h}.
\end{note}

