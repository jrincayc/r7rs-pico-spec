\extrapart{Examples}

Here are examples using Pico Scheme.

\begin{scheme}
(define list (lambda l l))
(list 'a 'b 'c)         \ev (a b c)
\end{scheme}
\begin{scheme}
(define list? (lambda (l)
  (cond ((null? l) \schtrue)
        ((not (pair? l)) \schfalse)
        (else (list? (cdr l)))
   )
))
(list? '(a b c))        \ev \schtrue
(list? (cons 'a 'b))    \ev \schfalse
\end{scheme}

Returns a list consisting of the elements of {\em l} followed by {\em t}

\begin{scheme}
(define append (lambda (l t)
  (cond ((null? l) t)
          (else (cons (car l) (append (cdr l) t))))
))
(append '() '(a))      \ev (a)
(append '(a b) '(c d)) \ev (a b c d)
\end{scheme}

This procedure returns the first sublist of {\em l} whose car is {\em obj}.

\begin{scheme}
(define assq (lambda (obj l)
  (cond ((null? l) \schfalse)
        ((eq? obj (car (car l))) (car l))
        (else (assq obj (cdr l)))
)))
(define e '((a 1) (b 2) (c 3)))
(assq 'a e)            \ev (a 1)
(assq 'b e)            \ev (b 2)
(assq 'd e)            \ev \schfalse
\end{scheme}

This shows using the Y combinator to create a recursive function.

\begin{scheme}
(let ((Y (lambda (phi)
           ((lambda (f) (f f))
            (lambda (f)
              (phi (lambda x (apply (f f) x))))))))
  (let ((fact
         (Y (lambda (fact)
              (lambda (n)
                (if (< n 2) 1
                    (* n (fact (- n 1)))))))))
    (fact 5))) \ev 120
\end{scheme}
