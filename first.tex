% First page

\thispagestyle{empty}


\topnewpage[{
\begin{center}   {\huge\bf
	DRAFT - Tiny Revised{\Huge$^{\mathbf{7}}$} Report on the Algorithmic Language 				Scheme}

\vskip 1ex
$$
\begin{tabular}{l@{\extracolsep{.5in}}l@{\extracolsep{.5in}}l}
\multicolumn{3}{c}{(Tiny subset extracted by J\authorsc{OSHUA} C\authorsc{OGLIATI})}\\
\multicolumn{3}{c}{A\authorsc{LEX} S\authorsc{HINN},
J\authorsc{OHN} C\authorsc{OWAN}, \authorsc{AND}
A\authorsc{RTHUR} A. G\authorsc{LECKLER} (\textit{Editors})} \\
\\
S\authorsc{TEVEN} G\authorsc{ANZ} &
A\authorsc{LEXEY} R\authorsc{ADUL} &
O\authorsc{LIN} S\authorsc{HIVERS} \\

A\authorsc{ARON} W. H\authorsc{SU} &
J\authorsc{EFFREY} T. R\authorsc{EAD} &
A\authorsc{LARIC} S\authorsc{NELL}-P\authorsc{YM} \\

B\authorsc{RADLEY} L\authorsc{UCIER} &
D\authorsc{AVID} R\authorsc{USH} &
G\authorsc{ERALD} J. S\authorsc{USSMAN} \\

E\authorsc{MMANUEL} M\authorsc{EDERNACH} &
B\authorsc{ENJAMIN} L. R\authorsc{USSEL} &
\\
\\
\multicolumn{3}{c}{R\authorsc{ICHARD} K\authorsc{ELSEY},
W\authorsc{ILLIAM} C\authorsc{LINGER},
\authorsc{AND} J\authorsc{ONATHAN} R\authorsc{EES}} \\
\multicolumn{3}{c}{\textit{(Editors, Revised$^{\mathit{5}}$ Report on the Algorithmic Language Scheme)}} \\
\\
\multicolumn{3}{c}{M\authorsc{ICHAEL} S\authorsc{PERBER},
R. K\authorsc{ENT} D\authorsc{YBVIG}, M\authorsc{ATTHEW} F\authorsc{LATT},
\authorsc{AND} A\authorsc{NTON} \authorsc{VAN} S\authorsc{TRAATEN}} \\
\multicolumn{3}{c}{\textit{(Editors, Revised$^{\mathit{6}}$ Report on the Algorithmic Language Scheme)}} \\
\end{tabular}
$$
\vskip 2ex
{\it Dedicated to the memory of John McCarthy and Daniel Weinreb}
\vskip 2.6ex
{\large \bf \today}             % *** DRAFT ***
\vskip 2ex
\end{center}
}]

%\clearpage
\vskip 2ex

\chapter*{Summary}

The report gives a defining description of the tiny subset of the programming language
Scheme.  Scheme is a statically scoped and properly tail recursive
dialect of the Lisp programming language~\cite{McCarthy} invented by Guy Lewis
Steele~Jr.\ and Gerald Jay~Sussman.  It was designed to have
exceptionally clear and simple semantics and few different ways to
form expressions.  Tiny Scheme is a purely functional subset of Scheme.

\vest The introduction offers a brief history of the language and of
the report.

\vest The first three chapters present the fundamental ideas of the
language and describe the notational conventions used for describing the
language and for writing programs in the language.

\vest Chapters~\ref{expressionchapter} and~\ref{programchapter} describe
the syntax and semantics of expressions, definitions, programs, and libraries.

\vest Chapter~\ref{builtinchapter} describes Scheme's built-in
procedures, which include all of the language's data manipulation primitives.

\vest Chapter~\ref{formalchapter} provides a formal syntax for Scheme
written in extended BNF, along with a formal denotational semantics.
An example of the use of the language follows the formal syntax and
semantics.


\vest The report concludes with references and an
index.

\begin{note}
The editors of the \rsevenrs, \rfivers\ and \rsixrs\ reports are
listed as authors of this report in recognition of the substantial
portions of this report that are copied directly from \rfivers, \rsixrs\ and \rsevenrs.
There is no intended implication that those editors, individually or
collectively, support or do not support this report.
\end{note}

\subsection*{Acknowledgments}

We would like to thank all the past editors of \rsevenrs\cite{R7RS}, and the
people who helped them in turn: Hal Abelson, Norman Adams, David
Bartley, Alan Bawden, Michael Blair, Gary Brooks, George Carrette,
Andy Cromarty, Pavel Curtis, Jeff Dalton, Olivier Danvy, Ken Dickey,
Bruce Duba, Robert Findler, Andy Freeman, Richard Gabriel, Yekta
G\"ursel, Ken Haase, Robert Halstead, Robert Hieb, Paul Hudak, Morry
Katz, Eugene Kohlbecker, Chris Lindblad, Jacob Matthews, Mark Meyer,
Jim Miller, Don Oxley, Jim Philbin, Kent Pitman, John Ramsdell,
Guillermo Rozas, Mike Shaff, Jonathan Shapiro, Guy Steele, Julie
Sussman, Perry Wagle, Mitchel Wand, Daniel Weise, Henry Wu, and Ozan
Yigit.  We thank Carol Fessenden, Daniel Friedman, and Christopher
Haynes for permission to use text from the Scheme 311 version 4
reference manual.  We thank Texas Instruments, Inc.~for permission to
use text from the {\em TI Scheme Language Reference
Manual}~\cite{TImanual85}.  We gladly acknowledge the influence of
manuals for MIT Scheme~\cite{MITScheme}, T~\cite{Rees84}, Scheme
84~\cite{Scheme84}, Common Lisp~\cite{CLtL}, and Algol 60~\cite{Naur63},
as well as the following SRFIs:  0, 1, 4, 6, 9, 11, 13, 16, 30, 34, 39, 43, 46, 62, and 87,
all of which are available at {\cf http://srfi.schemers.org}.



In addition, the subset of Scheme in
{\em The Little Schemer}~\cite{LittleSchemer} provided inspiration for
Tiny Scheme.


\todo{expand the summary so that it fills up the column.}

\vfill
\eject

\chapter*{Contents}
\addvspace{3.5pt}                  % don't shrink this gap
\renewcommand{\tocshrink}{-3.5pt}  % value determined experimentally
{\footnotesize
\tableofcontents
}

\vfill
\eject
