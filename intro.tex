\extrapart{Introduction}

\label{historysection}

Programming languages should be designed not by piling feature on top
of feature, but by removing the weaknesses and restrictions that make
additional features appear necessary.  Pico Scheme continues this
tradition by creating a smaller subset of \rsevenrs{} that can be easily
implemented and understood, yet remains a full programming
language. Features including side effects and continuations that add
complication to both the denotational semantics and the implementation
are removed. In this report \rsevenrs is used to refer to the ``small''
language.


\subsection*{Background}

\vest The first description of Scheme was written in
1975~\cite{Scheme75}.  A revised report~\cite{Scheme78}
appeared in 1978, which described the evolution
of the language as its MIT implementation was upgraded to support an
innovative compiler~\cite{Rabbit}. An introductory
computer science textbook using Scheme was published in
1984~\cite{SICP}.

\vest Fifteen representatives of the major implementations of Scheme
met in October 1984.
Their report, the RRRS~\cite{RRRS},
was published at MIT and Indiana University in the summer of 1985.
Further revision took place in the spring of 1986, resulting in the
\rthreers~\cite{R3RS}.
Work in the spring of 1988 resulted in \rfourrs~\cite{R4RS},
which became the basis for the
IEEE Standard for the Scheme Programming Language in 1991~\cite{IEEEScheme}.
In 1998, several additions to the IEEE standard, including high-level
hygienic macros, multiple return values, and {\cf eval}, were finalized
as the \rfivers~\cite{R5RS}.

In the fall of 2006, work began on a more ambitious standard.
The resulting standard, the
\rsixrs, was completed in August 2007~\cite{R6RS}.

In 2009 the Scheme Steering Committee decided to divide the
standard into two separate but compatible languages --- a ``small''
language and a ``large'' language.
The the ``small'' language of that effort resulted in \rsevenrs~\cite{R7RS}.


\medskip

We intend this report to belong to the entire Scheme community, and so
we grant permission to copy it in whole or in part without fee.  In
particular, we encourage implementers of Pico Scheme to use this report as
a starting point for manuals and other documentation, modifying it as
necessary.




