% Lexical structure

%%\vfill\eject
\chapter{Lexical conventions}

This section gives an informal account of some of the lexical
conventions used in writing Scheme programs.  For a formal syntax of
Scheme, see section~\ref{BNF}.

\section{Identifiers}
\label{syntaxsection}

An identifier\mainindex{identifier} is any sequence of letters, digits, and
``extended identifier characters'' provided that it does not have a prefix
which is a valid number.  
However, the  \ide{.} token (a single period) used in the list syntax
is not an identifier.

All implementations of Scheme must support the following extended identifier
characters:

\begin{scheme}
!\ \$ \% \verb"&" * + - . / :\ < = > ? @ \verb"^" \verb"_" \verb"~" %
\end{scheme}

Here are some examples of identifiers:

\begin{scheme}
...                      {+}
+soup+                   <=?
->string                 a34kTMNs
lambda                   list->vector
q                        V17a
the-word-recursion-has-many-meanings%
\end{scheme}


See section~\ref{identifiersyntax} for the formal syntax of identifiers.

\vest Identifiers have two uses within Scheme programs:
\begin{itemize}
\item Any identifier can be used as a variable\mainindex{variable}
 or as a syntactic keyword\mainindex{syntactic keyword}
(see section~\ref{variablesection}).

\item When an identifier appears as a literal or within a literal
(see section~\ref{quote}), it is being used to denote a {\em symbol}
(see section~\ref{symbolsection}).
\end{itemize}

In contrast with earlier revisions of the report~\cite{R5RS}, the
syntax distinguishes between upper and lower case in
identifiers and in characters specified using their names.
None of the identifiers defined in this report contain upper-case
characters, even when they appear to do so as a result
of the English-language convention of capitalizing the first word of
a sentence.


\section{Whitespace and comments}
\label{wscommentsection}

\defining{Whitespace} characters include the space, tab, and newline characters.
(Implementations may provide additional whitespace characters such
as page break.)  Whitespace is used for improved readability and
as necessary to separate tokens from each other, a token being an
indivisible lexical unit such as an identifier or number, but is
otherwise insignificant.  Whitespace can occur between any two tokens,
but not within a token.

The lexical syntax includes one comment form.
Comments are treated exactly like whitespace.

A semicolon ({\tt;}) indicates the start of a line
comment.\mainindex{comment}\mainschindex{;}  The comment continues to the
end of the line on which the semicolon appears.

\section{Other notations}

\todo{Rewrite?}

For a description of the notations used for numbers, see
section~\ref{numbersection}.

\begin{description}{}{}

\item[{\tt.\ + -}]
These are used in numbers, and can also occur anywhere in an identifier.
A delimited plus or minus sign by itself
is also an identifier.
Note that a sequence of two or more periods {\em is} an identifier.

\item[\tt( )]
Parentheses are used for grouping and to notate lists
(section~\ref{listsection}).

\item[\singlequote]
The apostrophe (single quote) character is used to indicate literal data (section~\ref{quote}).

% A box used because \verb is not allowed in command arguments.
\setbox0\hbox{\tt \verb"[" \verb"]" \verb"{" \verb"}"}
\item[\copy0]
Left and right square and curly brackets (braces)
are reserved for possible future extensions to the language.

\setbox0\hbox{\backquote \ \tt, ,@ \tt" \backwhack}
\item[\copy0]
The grave accent,
character comma and sequence comma at-sign, quotation mark and
backslash are used by \rsevenrs.

\item[\schtrue{} \schfalse{}]
  These are the boolean constants (section~\ref{booleansection}).

\end{description}
