% Initial environment

%\vfill\eject
\chapter{Standard procedures}
\label{initialenv}
\label{builtinchapter}

\mainindex{initial environment}
\mainindex{global environment}
\mainindex{procedure}

This chapter describes Scheme's built-in procedures.


A program can use a global variable definition to bind any variable.
These operations do not modify the behavior of
any procedure defined in this report. Altering any global binding that has
not been introduced by a definition has an unspecified effect on the
behavior of the procedures defined in this chapter.

\section{Equivalence predicates}
\label{equivalencesection}

A \defining{predicate} is a procedure that always returns a boolean
value (\schtrue{} or \schfalse).  An \defining{equivalence predicate} is
the computational analogue of a mathematical equivalence relation; it is
symmetric, reflexive, and transitive.


\begin{entry}{%
\proto{eqv?}{ \vari{obj} \varii{obj}}{procedure}}

The {\cf eqv?}\ procedure can determine if symbols, numbers and
booleans are equivalent. The empty list is only equivalent to another empty
list. Two different types are never equivalent, and other comparisons
are unspecified.

The {\cf eqv?} procedure returns \schtrue{} if:

\begin{itemize}
\item \vari{obj} and \varii{obj} are both \schtrue{} or both \schfalse.

\item \vari{obj} and \varii{obj} are both symbols and are the same
symbol (section~\ref{symbolsection}).

\item \vari{obj} and \varii{obj} are both numbers and
are numerically equal (in the sense of {\cf =}).

\item \vari{obj} and \varii{obj} are both the empty list.
\end{itemize}

The {\cf eqv?} procedure returns \schfalse{} if:

\begin{itemize}
\item \vari{obj} and \varii{obj} are of different types
(section~\ref{disjointness}).

\item one of \vari{obj} and \varii{obj} is \schtrue{} but the other is
\schfalse{}.

\item \vari{obj} and \varii{obj} are symbols but are not the same
symbol (section~\ref{symbolsection}).

\item \vari{obj} and \varii{obj} are both numbers and
are numerically unequal (in the sense of {\cf =}).

\item one of \vari{obj} and \varii{obj} is the empty list but the other
is not.

\end{itemize}


\begin{scheme}
(eqv? 'a 'a)                     \ev  \schtrue
(eqv? 'a 'b)                     \ev  \schfalse
(eqv? '(a) '(a))                 \ev  \unspecified
(eqv? (list 'a) (list 'a))       \ev  \unspecified
(eqv? '() '())                   \ev  \schtrue
(eqv? 2 2)                       \ev  \schtrue
(eqv? car car)                   \ev  \unspecified
(let ((n (+ 2 3)))
  (eqv? n n))      \ev  \schtrue
(let ((x '(a)))
  (eqv? x x))      \ev  \unspecified
(let ((x '()))
  (eqv? x x))      \ev  \schtrue
(let ((p (lambda (x) x)))
  (eqv? p p))      \ev  \unspecified
(eqv? \#f 'nil)                  \ev  \schfalse%
\end{scheme}

\begin{rationale} {\cf eqv?}\ can be used to compare atoms, and
other uses are left unspecified.
\end{rationale}

\end{entry}

\section{Numbers}
\label{numbersection}
\index{number}

\newcommand{\type}[1]{{\it#1}}
\newcommand{\tupe}[1]{{#1}}

It is important to distinguish between mathematical numbers, the
Scheme numbers that attempt to model them, the machine representations
used to implement the Scheme numbers, and notations used to write
numbers.  This report uses the types \type{number}, and \type{integer}
to refer to both mathematical numbers and Scheme numbers.

Pico Scheme implementations should support integers sufficiently large
to calculate the length of any allowable list, and for binary computers that usually can
be satisfied by signed integers of the same bit length as machine
addresses.


\subsection{Syntax of numerical constants}
\label{numbernotations}

The syntax of the written representations for numbers is described formally in
section~\ref{numbersyntax}. Numbers are written in decimal.

\subsection{Numerical operations}

The reader is referred to section~\ref{typeconventions} for a summary
of the naming conventions used to specify restrictions on the types of
arguments to numerical routines.

\begin{entry}{%
\proto{number?}{ obj}{procedure}}

This numerical type predicate can be applied to any kind of
argument, including non-numbers.  It return \schtrue{} if the object is
of the named type, and otherwise it return \schfalse{}.

\begin{scheme}
(number? 3)         \ev  \schtrue
(number? '(1))      \ev  \schfalse%
\end{scheme}

\end{entry}

\begin{entry}{%
\proto{=}{ \vri{n} \vrii{n}}{procedure}
\proto{<}{ \vri{n} \vrii{n}}{procedure}
\proto{>}{ \vri{n} \vrii{n}}{procedure}}

These procedures return \schtrue{} if their arguments are (respectively):
equal, monotonically increasing, monotonically decreasing,
and \schfalse{} otherwise.

These predicates are required to be transitive.

\end{entry}

\begin{entry}{%
\proto{zero?}{ \vr{n}}{procedure}}

This numerical predicate tests if a number equals zero.

\end{entry}

\begin{entry}{%
\proto{+}{ \vri{n} \vrii{n}}{procedure}
\proto{*}{ \vri{n} \vrii{n}}{procedure}}

These procedures return the sum or product of their arguments.

\begin{scheme}
(+ 3 4)                 \ev  7
(* 4 5)                 \ev  20%
\end{scheme} 
 
\end{entry}

\begin{entry}{%
\proto{-}{ \vr{n}}{procedure}
\rproto{-}{ \vri{n} \vrii{n}}{procedure}}

With two arguments, this procedure returns the difference of its arguments, associating to the left.  With one argument,
however, it returns the additive inverse of its argument.

\begin{scheme}
(- 3 4)                 \ev  -1
(- 3)                   \ev  -3%
\end{scheme}

\end{entry}

\section{Booleans}
\label{booleansection}

The standard boolean objects for true and false are written as
\schtrue{} and \schfalse.\sharpindex{t}\sharpindex{f}  
What really
matters, though, are the objects that the Scheme conditional expressions
({\cf if}, {\cf cond}, {\cf and}, {\cf or}) treat as
true\mainindex{true} or false\mainindex{false}.  The phrase ``a true value''\mainindex{true}
(or sometimes just ``true'') means any object treated as true by the
conditional expressions, and the phrase ``a false value''\mainindex{false} (or
``false'') means any object treated as false by the conditional expressions.

\vest Of all the Scheme values, only \schfalse{}
counts as false in conditional expressions.
All other Scheme values, including \schtrue,
count as true.

\begin{note}
Unlike some other dialects of Lisp,
Scheme distinguishes \schfalse{} and the empty list \mainindex{empty list}
from each other and from the symbol {\cf nil}.
\end{note}

\vest Boolean constants evaluate to themselves, so they do not need to be quoted
in programs.

\begin{scheme}
\schtrue         \ev  \schtrue
\schfalse        \ev  \schfalse
'\schfalse       \ev  \schfalse%
\end{scheme}


\begin{entry}{%
\proto{not}{ obj}{procedure}}

The {\cf not} procedure returns \schtrue{} if \var{obj} is false, and returns
\schfalse{} otherwise.

\begin{scheme}
(not \schtrue)   \ev  \schfalse
(not 3)          \ev  \schfalse
(not '(3))   \ev  \schfalse
(not \schfalse)  \ev  \schtrue
(not '())        \ev  \schfalse
(not 'nil)       \ev  \schfalse%
\end{scheme}

\end{entry}


\begin{entry}{%
\proto{boolean?}{ obj}{procedure}}

The {\cf boolean?} predicate returns \schtrue{} if \var{obj} is either \schtrue{} or
\schfalse{} and returns \schfalse{} otherwise.

\begin{scheme}
(boolean? \schfalse)  \ev  \schtrue
(boolean? 0)          \ev  \schfalse
(boolean? '())        \ev  \schfalse%
\end{scheme}

\end{entry}

\section{Pairs and lists}
\label{listsection}

A \defining{pair} (sometimes called a \defining{dotted pair}) is a
record structure with two fields called the car and cdr fields (for
historical reasons).  Pairs are created by the procedure {\cf cons}.
The car and cdr fields are accessed by the procedures {\cf car} and
{\cf cdr}.

Pairs are used primarily to represent lists.  A \defining{list} can
be defined recursively as either the empty list\mainindex{empty list} or a pair whose
cdr is a list.  More precisely, the set of lists is defined as the smallest
set \var{X} such that

\begin{itemize}
\item The empty list is in \var{X}.
\item If \var{list} is in \var{X}, then any pair whose cdr field contains
      \var{list} is also in \var{X}.
\end{itemize}

The objects in the car fields of successive pairs of a list are the
elements of the list.  For example, a two-element list is a pair whose car
is the first element and whose cdr is a pair whose car is the second element
and whose cdr is the empty list.  The length of a list is the number of
elements, which is the same as the number of pairs.

The empty list\mainindex{empty list} is a special object of its own type.
It is not a pair, it has no elements, and its length is zero.

\begin{note}
The above definitions imply that all lists have finite length and are
terminated by the empty list.
\end{note}

The most general notation (external representation) for Scheme pairs is
the ``dotted'' notation \hbox{\cf (\vari{c} .\ \varii{c})} where
\vari{c} is the value of the car field and \varii{c} is the value of the
cdr field.  For example {\cf (4 .\ 5)} is a pair whose car is 4 and whose
cdr is 5.  Note that {\cf (4 .\ 5)} is the external representation of a
pair, not an expression that evaluates to a pair.

A more streamlined notation can be used for lists: the elements of the
list are simply enclosed in parentheses and separated by spaces.  The
empty list\mainindex{empty list} is written {\tt()}.  For example,

\begin{scheme}
(a b c d e)%
\end{scheme}

and

\begin{scheme}
(a . (b . (c . (d . (e . ())))))%
\end{scheme}

are equivalent notations for a list of symbols.

A chain of pairs not ending in the empty list is called an
\defining{improper list}.  Note that an improper list is not a list.
The list and dotted notations can be combined to represent
improper lists:

\begin{scheme}
(a b c . d)%
\end{scheme}

is equivalent to

\begin{scheme}
(a . (b . (c . d)))%
\end{scheme}

Whether a given pair is a list depends upon what is stored in the cdr
field.

Within literal expressions and representations of objects the form \singlequote\hyper{datum}\schindex{'} denotes a two-ele\-ment list whose first elements is
the symbols \ide{quote}.  The second element in each case
is \hyper{datum}.  This convention is supported so that arbitrary Scheme
programs can be represented as lists.  
That is, according to Scheme's grammar, every
\meta{expression} is also a \meta{datum} (see section~\ref{datum}).
See section~\ref{externalreps}. 


\begin{entry}{%
\proto{pair?}{ obj}{procedure}}

The {\cf pair?} predicate returns \schtrue{} if \var{obj} is a pair, and otherwise
returns \schfalse.

\begin{scheme}
(pair? '(a . b))        \ev  \schtrue
(pair? '(a b c))        \ev  \schtrue
(pair? '())             \ev  \schfalse
\end{scheme}
\end{entry}


\begin{entry}{%
\proto{cons}{ \vari{obj} \varii{obj}}{procedure}}

Returns a newly allocated pair whose car is \vari{obj} and whose cdr is
\varii{obj}. 

\begin{scheme}
(cons 'a '())           \ev  (a)
(cons '(a) '(b c d))    \ev  ((a) b c d)
(cons 'a 3)             \ev  (a . 3)
(cons '(a b) 'c)        \ev  ((a b) . c)%
\end{scheme}
\end{entry}


\begin{entry}{%
\proto{car}{ pair}{procedure}}

Returns the contents of the car field of \var{pair}.  Note that it is an
error to take the car of the empty list\mainindex{empty list}.

\begin{scheme}
(car '(a b c))          \ev  a
(car '((a) b c d))      \ev  (a)
(car '(1 . 2))          \ev  1
(car '())               \ev  \scherror%
\end{scheme}
 
\end{entry}


\begin{entry}{%
\proto{cdr}{ pair}{procedure}}

Returns the contents of the cdr field of \var{pair}.
Note that it is an error to take the cdr of the empty list.

\begin{scheme}
(cdr '((a) b c d))      \ev  (b c d)
(cdr '(1 . 2))          \ev  2
(cdr '())               \ev  \scherror%
\end{scheme}
 
\end{entry}

\begin{entry}{%
\proto{null?}{ obj}{procedure}}

Returns \schtrue{} if \var{obj} is the empty list\mainindex{empty list},
otherwise returns \schfalse.

\end{entry}

\section{Symbols}
\label{symbolsection}

Symbols are objects whose usefulness rests on the fact that two
symbols are identical (in the sense of {\cf eqv?}) if and only if their
names are spelled the same way.  For instance, they can be used
the way enumerated values are used in other languages.

\vest The rules for writing a symbol are exactly the same as the rules for
writing an identifier; see sections~\ref{syntaxsection}
and~\ref{identifiersyntax}.

\begin{entry}{%
\proto{symbol?}{ obj}{procedure}}

Returns \schtrue{} if \var{obj} is a symbol, otherwise returns \schfalse.

\begin{scheme}
(symbol? 'foo)          \ev  \schtrue
(symbol? (car '(a b)))  \ev  \schtrue
(symbol? 'nil)          \ev  \schtrue
(symbol? '())           \ev  \schfalse
(symbol? \schfalse)     \ev  \schfalse%
\end{scheme}
\end{entry}

\section{Control features}
\label{proceduresection}

\begin{entry}{%
\proto{procedure?}{ obj}{procedure}}

Returns \schtrue{} if \var{obj} is a procedure, otherwise returns \schfalse.

\begin{scheme}
(procedure? car)            \ev  \schtrue
(procedure? 'car)           \ev  \schfalse
(procedure? (lambda (x) (* x x)))   
                            \ev  \schtrue
(procedure? '(lambda (x) (* x x)))  
                            \ev  \schfalse
\end{scheme}

\end{entry}

\begin{entry}{%
\proto{apply}{ proc args}{procedure}}

The {\cf apply} procedure calls \var{proc} with the elements of the list
\var{args} as the actual
arguments.

\begin{scheme}
(apply + '(3 4))              \ev  7

(define compose
  (lambda (f g)
    (lambda args
      (f (apply g args)))))

((compose - *) 3 4)              \ev  -12%
\end{scheme}
\end{entry}
