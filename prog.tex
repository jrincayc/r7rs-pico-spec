\chapter{Program structure}
\label{programchapter}

\section{Programs}

A Scheme program consists of a sequence of
expressions, definitions, and output.
Expressions are described in chapter~\ref{expressionchapter}.
Definitions are variable definitions which are explained in this chapter.
They are valid in the outermost level of a \hyper{program}
and at the beginning of a \hyper{body}.
\mainindex{definition}

Expressions occurring at the outermost level of a program
do not create any bindings.  They are
executed in order when the program is
invoked or loaded, and typically perform some kind of initialization.

Programs are typically stored in files, although
in some implementations they can be entered interactively into a running
Scheme system.  Other paradigms are possible.

\begin{note}
  In order to simplify Pico Scheme, library declarations and importing
  are not part of the specification and not required.  However,
  \rsevenrs\ requires that programs start with one or more import
  declarations. Pico Scheme programs being run by \rsevenrs\ will need
  to have a {\cf (import (scheme base) (scheme write))} added at the
  start.

  Pico Scheme implementations may choose to implement \rsevenrs\ {\cf
    import}, {\cf define-library}, and {\cf export} directly as
  declarations.  A possible alternative implementation is adding a
  variables declaration (such as {\cf define-vars}) that uses the
  auxiliary function {\it extends} and appropriate rewrite rules to
  map {\cf import} to {\cf define-vars}.
\end{note}

\section{Variable definitions}
\label{defines}
\mainindex{variable definition}

A variable definition binds one identifier and specifies a
value for it.
The only kind of variable definition
takes the following form:\mainschindex{define}

\begin{itemize}

\item{\tt(define \hyper{variable} \hyper{expression})}

\end{itemize}

\subsection{Top level definitions}

At the outermost level of a program, a definition
\begin{scheme}
(define \hyper{variable} \hyper{expression})%
\end{scheme}
adds or updates the environment with the new assignment. Note
that the environment of a lambda expression used with define
includes the variable so
it can be called recursively.

\begin{scheme}
(define add3
  (lambda (x) (+ x 3)))
(add3 3)                            \ev  6
(define first car)
(first '(1 2))                      \ev  1%
\end{scheme}

\subsection{Internal definitions}
\label{internaldefines}

Definitions can occur at the
beginning of a \hyper{body} (that is, the body of a \ide{lambda}
or \ide{let}).
Such definitions are known as {\em internal definitions}\mainindex{internal
definition} as opposed to the global definitions described above.
The variables defined by internal definitions are local to the
\hyper{body}.  That is, \hyper{variable} is bound,
and the region of the binding is the following definitions and expressions in the \hyper{body}.  For example,

\begin{scheme}
(let ((x 5))
  (define bar (lambda (a b) (+ (* a b) a)))
  (define foo (lambda (y) (bar x y)))
  (foo (+ x 3)))                \ev  45%
\end{scheme}

In \rsevenrs{} it is an error to define the same identifier more than once in the
same \hyper{body}.

\begin{note}
Unlike \rsevenrs, in Pico scheme, the region binding of a definitions
is the following definitions and expressions in the \hyper{body}, not
the entire \hyper{body} due to differences in semantics from the
removal of {\cf set!}.
\end{note}

\section{The REPL}

Implementations may provide an interactive session called a
\defining{REPL} (Read-Eval-Print Loop), where
expressions and definitions can be
entered and evaluated one at a time.

An implementation may provide a mode of operation in which the REPL
reads its input from a file.

