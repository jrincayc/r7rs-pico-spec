\chapter{Program structure}
\label{programchapter}

\section{Programs}

A Scheme program consists of a sequence of
expressions, definitions and output.
Expressions are described in chapter~\ref{expressionchapter}.
Definitions are variable definitions which are explained in this chapter.
They are valid in some, but not all, contexts where expressions
are allowed, specifically at the outermost level of a \hyper{program}
and at the beginning of a \hyper{body}.
\mainindex{definition}

Expressions occurring at the outermost level of a program
do not create any bindings.  They are
executed in order when the program is
invoked or loaded, and typically perform some kind of initialization.

Programs are typically stored in files, although
in some implementations they can be entered interactively into a running
Scheme system.  Other paradigms are possible.

\section{Variable definitions}
\label{defines}
\mainindex{variable definition}

A variable definition binds one identifier and specifies an initial
value for it.
The only kind of variable definition
takes the following form:\mainschindex{define}

\begin{itemize}

\item{\tt(define \hyper{variable} \hyper{expression})}

\end{itemize}

\subsection{Top level definitions}

At the outermost level of a program, a definition
\begin{scheme}
(define \hyper{variable} \hyper{expression})%
\end{scheme}
which adds or updates the environment with the new assignment. Note
that the environment of a lambda expression includes the variable so
it can be called recursively.

\begin{scheme}
(define add3
  (lambda (x) (+ x 3)))
(add3 3)                            \ev  6
(define first car)
(first '(1 2))                      \ev  1%
\end{scheme}

\subsection{Internal definitions}
\label{internaldefines}

Definitions can occur at the
beginning of a \hyper{body} (that is, the body of a \ide{lambda},
or \ide{let}).  Note that
such a body might not be apparent until after expansion of other syntax.
Such definitions are known as {\em internal definitions}\mainindex{internal
definition} as opposed to the global definitions described above.
The variables defined by internal definitions are local to the
\hyper{body}.  That is, \hyper{variable} is bound rather than assigned,
and the region of the binding is the following definitions and expressions in the \hyper{body}.  For example,

\begin{scheme}
(let ((x 5))
  (define bar (lambda (a b) (+ (* a b) a)))
  (define foo (lambda (y) (bar x y)))
  (foo (+ x 3)))                \ev  45%
\end{scheme}

It is an error if it is not
possible to evaluate each \hyper{expression} of every internal
definition in a \hyper{body} without assigning or referring to
the value of the corresponding \hyper{variable} or the \hyper{variable}
of any of the definitions that follow it in \hyper{body}.

It is an error to define the same identifier more than once in the
same \hyper{body}.

\section{The REPL}

Implementations may provide an interactive session called a
\defining{REPL} (Read-Eval-Print Loop), where
expressions and definitions can be
entered and evaluated one at a time.

An implementation may provide a mode of operation in which the REPL
reads its input from a file.

